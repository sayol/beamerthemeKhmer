\documentclass[serif]{beamer}
\usetheme{Khmer}
\colorlet{khmer}{blue}
\title[\LaTeX]{ភាសា \LaTeX}
\author[សាយ]{អូល~សាយ}
\institute[វ.គ.ភ]{វិទ្យាស្ថានគរុកោសល្យរាជធានីភ្នំពេញ}
\date[\khmershortdate]{\khmerdate}
\begin{document}
    \section{ស្វាគមន៍}
    \begin{frame}
        \titlepage
    \end{frame}
    \section{លេខ និងអក្សរខ្មែរ}
    \begin{frame}
        \begin{definition}
            \begin{enumerate}[m]
                \item ជាន់ទីមួយ
                \begin{enumerate}[k]
                    \item ជាន់ទីពីរ
                    \begin{enumerate}[i]
                        \item ជាន់ទីបី
                        \item ជាន់ទីបី
                    \end{enumerate}
                    \item ជាន់ទីពីរ
                \end{enumerate}
                \item ជាន់ទីមួយ
            \end{enumerate}
        \end{definition}
    \end{frame}
    \section{លាយបញ្ចូលគ្នា}
    \begin{frame}
        \begin{theorem}
            \begin{enumerate}[A]
            \item ជាន់ទីមួយ
            \begin{enumerate}[1]
                \item ជាន់ទីពីរ
                \begin{enumerate}[i]
                    \item ជាន់ទីបី
                    \item ជាន់ទីបី
                \end{enumerate}
                \item ជាន់ទីពីរ
            \end{enumerate}
            \item ជាន់ទីមួយ
        \end{enumerate}
        \end{theorem}
    \end{frame}
    \begin{frame}
        \begin{example}
            \begin{enumerate}[<+-|alert@+>][I]
                \item ជាន់ទីមួយ
                \begin{enumerate}[1]
                    \item ជាន់ទីពីរ
                    \begin{enumerate}[a]
                        \item ជាន់ទីបី
                        \item ជាន់ទីបី
                    \end{enumerate}
                    \item ជាន់ទីពីរ
                \end{enumerate}
                \item ជាន់ទីមួយ
            \end{enumerate}
        \end{example}
    \end{frame}
\section{បរិស្ថានផ្សេងៗ}
\begin{frame}
    \begin{problem}
        គណនាផលបូកស៊េរី $ \sum_{n=1}^{\infty} \frac{1}{n(n+1)} $~។
    \end{problem}
    \begin{proof}
        ចំពោះគ្រប់ $ n\in\mathbb{N} $ គេបាន $ \frac{1}{n(n+1)}=\frac{1}{n}-\frac{1}{n+1} $ នាំឱ្យ
        \begin{align*}
        \sum_{n=1}^{\infty} \frac{1}{n(n+1)}
        &=\lim\limits_{n\to \infty} \sum_{k=1}^{n} \left( \frac{1}{k}-\frac{1}{k+1} \right)\\
        &=\lim\limits_{n\to \infty} \left( 1-\frac{1}{n+1} \right)\\
        &=1
        \end{align*}
    \end{proof}
\end{frame}
\section{ឯកសារយោង}
\begin{frame}
    \begin{thebibliography}{4}
        \bibitem[tau,2018]{tau13} Tau, The beamer class,
        \newblock User Guide for version 3.55.
        \setbeamertemplate{bibliography item}[book]
        \bibitem[knuth,1986]{knuth86} Knuth, The TeXbook, 1986
        \setbeamertemplate{bibliography item}[online]
        \bibitem[wikibook]{wikibook} wikibook, \url{www.wikibook.com/latex/}
    \end{thebibliography}
\end{frame}
\end{document}