\documentclass[12pt,serif]{beamer}
\usetheme{Khmer}
\begin{document}
    \title[\LaTeX]{ភាសា \LaTeX}
    \author[សាយ]{អូល~សាយ}
    \institute[វ.គ.ភ]{វិទ្យាស្ថានគរុកោសល្យរាជធានីភ្នំពេញ}
    \date[០២/២៣/២០]{២៣ កុម្ភៈ ២០២០}
    \section{ស្វាគមន៍}
    \begin{frame}{ស្វាគមន៍}
        \titlepage
    \end{frame}
    \section{លេខ និងអក្សរខ្មែរ}
    \begin{frame}{លេខ និងអក្សរខ្មែរ}
        \begin{definition}
            \begin{enumerate}[m]
                \item ជាន់ទីមួយ
                \begin{enumerate}[k]
                    \item ជាន់ទីពីរ
                    \begin{enumerate}[i]
                        \item ជាន់ទីបី
                        \item ជាន់ទីបី
                    \end{enumerate}
                    \item ជាន់ទីពីរ
                \end{enumerate}
                \item ជាន់ទីមួយ
            \end{enumerate}
        \end{definition}
    \end{frame}
    \section{លាយបញ្ចូលគ្នា}
    \begin{frame}{លាយបញ្ចូលគ្នា}
        \begin{theorem}
            \begin{enumerate}[A]
            \item ជាន់ទីមួយ
            \begin{enumerate}[1]
                \item ជាន់ទីពីរ
                \begin{enumerate}[i]
                    \item ជាន់ទីបី
                    \item ជាន់ទីបី
                \end{enumerate}
                \item ជាន់ទីពីរ
            \end{enumerate}
            \item ជាន់ទីមួយ
        \end{enumerate}
        \end{theorem}
    \end{frame}
    \begin{frame}{លាយបញ្ចូលគ្នា}
        \begin{example}
            \begin{enumerate}[I]
                \item ជាន់ទីមួយ
                \begin{enumerate}[1]
                    \item ជាន់ទីពីរ
                    \begin{enumerate}[a]
                        \item ជាន់ទីបី
                        \item ជាន់ទីបី
                    \end{enumerate}
                    \item ជាន់ទីពីរ
                \end{enumerate}
                \item ជាន់ទីមួយ
            \end{enumerate}
        \end{example}
    \end{frame}
\end{document}